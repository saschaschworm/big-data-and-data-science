% !TeX spellcheck = en_US

% Initialization
\documentclass[8pt, aspectratio=169, notes]{beamer}
\usetheme[sectionpage=simple, subsectionpage=simple, numbering=counter]{metropolis}

% Layout and Graphic Packages
\usepackage{array}
\usepackage{layouts}
\usepackage{booktabs}
\usepackage[export]{adjustbox}
\usepackage{caption}
\usepackage{setspace}
\usepackage{color, colortbl}
\usepackage[normalem]{ulem}
\usepackage{multicol}
\usepackage{ragged2e}
\usepackage{tikz}

% Math and Source Code Packages
\usepackage[ruled, lined, linesnumbered]{algorithm2e}
\usepackage{listings}
\usepackage{mathtools}

% Miscellaneous Packages
\usepackage{subfiles}
\usepackage{hyperref}

% Feature Flag Initialization
\newif\iffom

% Token Initialization
\newtoks\currentsemester
\newtoks\presenters
\newtoks\gitcommit
\newtoks\gitdate

% Step Counter Initialization
\newcounter{openexercise}
\newcounter{devexercise}

% Default Feature Flags
\fomtrue

% Default Tokens
\currentsemester{Winter Term 2019/2020}
\presenters{Sascha Schworm, M.Sc.}
\gitcommit{}
\gitdate{}

\iffom 
    \titlegraphic{\hfill\includegraphics[height=1cm]{assets/logos/fom.eps}}
    \institute{FOM University of Applied Science \\ Business Informatics \\\\ \the\currentsemester}
\else
    \titlegraphic{\hfill\includegraphics[height=1cm]{assets/logos/ssbe.eps}}
    \institute{Chair of Finance and Corporate Governance \\ Chair of Economic Statistics and Econometrics \\ Schumpeter School of Business and Economics \\ University of Wuppertal \\\\ \the\currentsemester}
\fi

% Colors Definitions
\definecolor{LightRed}{rgb}{1,0.88,0.88}
\definecolor{LightGreen}{rgb}{0.88,1,0.88}

% Code Style Definition
\lstdefinestyle{material}{basicstyle=\color[HTML]{2B2B2B}\small\ttfamily, keywordstyle=\color[HTML]{CC7832}, stringstyle=\color[HTML]{6A8759}, commentstyle=\color[HTML]{808080}, numberstyle=\color{gray}, frame=single, framerule=.5pt, rulecolor=\color{gray}, numbers=left, stepnumber=1, numbersep=5pt, showspaces=false, showstringspaces=false, showtabs=false, tabsize=4, breaklines=true, breakatwhitespace=true, morekeywords={None}, keywords=[2]{self}, keywordstyle=[2]\color[HTML]{94558D}, keywords=[3]{len, int, object}, keywordstyle=[3]\color[HTML]{8888C6}, keywords=[4]{__init__}, keywordstyle=[4]\color[HTML]{B200B2}, framexleftmargin=0pt, xleftmargin=3pt, linewidth=\dimexpr\linewidth}

% Beamer Temlate Customization
\setbeamertemplate{frame footer}{\the\presenters}
\setbeamertemplate{section in toc}[sections numbered]
\setbeamertemplate{subsection in toc}[subsections numbered]

\newsavebox{\mytempbox}
\setbeamertemplate{note page}
{
    {
        \scriptsize
        \usebeamerfont{note title}\usebeamercolor[fg]{note title}
        \ifbeamercolorempty[bg]{note title}{}{
            \insertvrule{.25\paperheight}{note title.bg}
            \vskip-.25\paperheight
            \nointerlineskip
        }
        \vbox{
            \hspace*{.6875\paperwidth}\insertslideintonotes{0.25}
            \vskip-0.25\paperheight
            \nointerlineskip
        }
        \nointerlineskip
        \vbox to .25\paperheight{\vskip1.5em
            \hbox{\insertshorttitle[width=8cm]}
            \setbox\mytempbox=\hbox{\insertsection}
            \hbox{\ifdim\wd\mytempbox>1pt{\hskip4pt\raise3pt\hbox{\vrule
                        width0.4pt height7pt\vrule width 9pt
                        height0.4pt}}\hskip1pt\hbox{\begin{minipage}[t]{7.5cm}\def\breakhere{}\insertsection\end{minipage}}\fi
            }
            \setbox\mytempbox=\hbox{\insertsubsection}
            \hbox{\ifdim\wd\mytempbox>1pt{\hskip17.4pt\raise3pt\hbox{\vrule
                        width0.4pt height7pt\vrule width 9pt
                        height0.4pt}}\hskip1pt\hbox{\begin{minipage}[t]{7.5cm}\def\breakhere{}\insertsubsection\end{minipage}}\fi
            }
            \setbox\mytempbox=\hbox{\insertshortframetitle}
            \hbox{\ifdim\wd\mytempbox>1pt{\hskip30.8pt\raise3pt\hbox{\vrule
                        width0.4pt height7pt\vrule width 9pt
                        height0.4pt}}\hskip1pt\hbox{\insertshortframetitle[width=7cm]}\fi
            }
            \vfil}
    }
    \vskip1em
    \nointerlineskip
    \textwidth=13.8cm
    \hsize=\textwidth
    \normalsize
    \insertnote
}

% Title Page Configuration
\title{Big Data \& Data Science}
\author{\textbf{\the\presenters}}
\date{}

% Beamer Configuration
\setbeamercolor{part title}{fg=black!2, bg=mDarkTeal}
\setbeamerfont{part title}{size=\Huge}

% Caption Configuration
\captionsetup[figure]{font={stretch=0.8}}

% Hyperref Configuration
\hypersetup{colorlinks=true, linkcolor=, urlcolor=cyan}

\begin{document}
    \maketitle
    
    \begin{frame}{Introduction to Big Data \& Data Science}
        \begin{alertblock}{\textsc{License}}
            These lecture slides as well as accompanying course materials such as source codes and notebooks have been developed by \href{https://www.saschaschworm.de}{Sascha Schworm} at the \href{https://www.wiwi.uni-wuppertal.de/}{University of Wuppertal}. Except where otherwise noted, all contents are licensed under a \href{https://creativecommons.org/licenses/by-nc-sa/4.0/}{Creative Commons BY-NC-SA 4.0} license.
        \end{alertblock}
        \begin{alertblock}{\textsc{Contribution}}
            If you find any mistakes in the lecture slides and/or accompanying course materials, feel free to either post an issue at the related \href{https://github.com/saschaschworm/big-data-and-data-science}{GitHub repository} or fork the repository, fix the mistakes by yourself and create a pull request. Any help is appreciated and benefits future students participating the course.
        \end{alertblock}
        \begin{alertblock}{\textsc{Version}}
            \the\gitcommit~(\the\gitdate)
        \end{alertblock}
    \end{frame}

    \section{Preliminaries}
    
%    \subfile{sections/preliminaries.tex}
    
    \part{Introduction to Data Science and Machine Learning}
    
    \section{Data Science Fundamentals}
     
%    \subfile{sections/data-science-fundamentals.tex}
    
    \section{Machine Learning Fundamentals}
    
%    \subfile{sections/machine-learning-fundamentals.tex}

    \section{Exploratory Data Analysis}
    
%    \subfile{sections/exploratory-data-analysis.tex}
    
    \part{Introduction to Data Science and Machine Learning}
    
    \section{Regression Models}
    
%    \subfile{sections/regression-models.tex}
    
    \section{Decision Tree Models}
    
%    \subfile{sections/decision-tree-models.tex}
       
    \section{Support Vector Machines}
    
%    \subfile{sections/support-vector-machines.tex}
    
    \section{Anomaly Detection}
    
%    \subfile{sections/anomaly-detection.tex}
    
    \section{Artificial Neural Networks}
    
%    \subfile{sections/artificial-neural-networks.tex}
\end{document}